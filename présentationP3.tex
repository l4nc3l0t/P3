% !TEX root = PSanté_03_présentation.tex PrésentationDiapoP3.tex

\usefonttheme{serif}
\usepackage{fontspec}
	\setmainfont{TeX Gyre Heros}
\usepackage{unicode-math}
\usepackage{lualatex-math}
	\setmathfont{TeX Gyre Termes Math}
\usepackage{polyglossia}
\setdefaultlanguage[frenchpart=false]{french}
\setotherlanguage{english}
\usepackage{microtype}
\usepackage[locale = FR,
            separate-uncertainty,
            multi-part-units = single,
            range-units = single]{siunitx}
	\DeclareSIUnit\an{an}
  \DeclareSIUnit{\octet}{o}
\usepackage{amsmath}
\usepackage{amsfonts}
\usepackage{amssymb}
\usepackage{array}
\usepackage{graphicx}
\graphicspath{{./Figures/}}
\usepackage{booktabs}
\usepackage{tabularx}
\usepackage{multirow}
\usepackage{multicol}
    \newcolumntype{L}{>{\raggedright\arraybackslash}X}
    \newcolumntype{R}{>{\raggedleft\arraybackslash}X}
\usepackage{tikz}
\usetikzlibrary{mindmap}
\usetikzlibrary{overlay-beamer-styles}
\usepackage{subcaption}
\usepackage[]{animate}
\usepackage{float}
\usepackage{csquotes}

\usetheme[secheader
         ]{Boadilla}
\usecolortheme{seagull}
\setbeamertemplate{enumerate items}[default]
\setbeamertemplate{itemize items}{-}
\setbeamertemplate{navigation symbols}{}
\setbeamertemplate{bibliography item}{}
\setbeamerfont{framesubtitle}{size=\large}
\setbeamertemplate{section in toc}[sections numbered]
%\setbeamertemplate{subsection in toc}[subsections numbered]

\title[Concevez une application au service de la santé publique]{Projet 3 : Concevez une application au service de la santé publique}
\author[Lancelot \textsc{Leclercq}]{Lancelot \textsc{Leclercq}} 
\institute[]{}
\date[]{\small{novembre 2021}}

%\AtBeginSection[]{
%  \begin{frame}
%  \vfill
%  \centering
%    \usebeamerfont{title}\insertsectionhead\par%
%  \vfill
%  \end{frame}
%}

\begin{document}
\setbeamercolor{background canvas}{bg=gray!20}
\begin{frame}[plain]
  \titlepage
\end{frame}

\section{Introduction}

\subsection{Problématique}
\begin{frame}{\insertsubsection}
  \begin{columns}
    \begin{column}{.6\textwidth}
      \begin{itemize}
        \item Application au service de la santé publique
        \item[]
        \item Idées d'application en lien avec l'alimentation
      \end{itemize}
    \end{column}
    \begin{column}{.4\textwidth}
      \includegraphics[width=.8\textwidth]{./Sante-publique-France-logo.pdf}
    \end{column}
  \end{columns}
\end{frame}

\subsection{Jeu de données}
\begin{frame}{\insertsubsection}
  \begin{columns}
    \begin{column}{.6\textwidth}
      \begin{itemize}
        \item Jeu de données issu de Open Food Facts
        \item[]
        \item Base de donnée ouverte et participative
        \item[]
        \item Repertorie les ingrédients, les allergènes, composition nutritionnelle, les labels, l'origine, etc
        \item[]
        \item Grosse base de données (environ \qty{4.5}{\giga\octet})
      \end{itemize}
    \end{column}
    \begin{column}{.4\textwidth}
      \centering
      \includegraphics[width=.8\textwidth]{./openfoodfacts-logo-en.pdf}
    \end{column}
  \end{columns}
\end{frame}


\section{Nettoyage du jeu de données}
\subsection{Produits vendus en France}
\begin{frame}{\insertsubsection}
  \begin{columns}[T]
    \begin{column}{.4\textwidth}
      \begin{itemize}
        \item Sélection des produits vendus en France
        \item \num{850000} entrées sur les 2 milliards d'origine
      \end{itemize}
    \end{column}
    \begin{column}{.6\textwidth}
      \begin{itemize}
        \item Suppression des colonnes dans lesquelles on a moins de 1\% de données
              et des colonnes contenant des tags et des versions anglaises qui sont redondantes avec les
              colonnes du mêmes nom
        \item 57 colonnes restantes sur les 186 d'origine
      \end{itemize}
    \end{column}
  \end{columns}
  \vfill
  \begin{figure}
    \includegraphics[width=.75\textwidth]{DataDropFinal.pdf}
  \end{figure}
\end{frame}

\subsection{Nettoyage des outliers}
\begin{frame}{\insertsubsection}
  \begin{columns}
    \begin{column}{.3\textwidth}
      \begin{itemize}
        \item Valeurs extrèmes pour :
              \begin{itemize}
                \item Quantité servie
                \item Empreinte carbone
                \item Énergie
              \end{itemize}
        \item[]
        \item Des valeurs très importantes là où le nombre de valeurs est très faible
        \item[]
        \item Nous écartons ces valeurs lorsqu'elles sont en dehors du quartile qui contient 99\% des valeurs.
      \end{itemize}
    \end{column}
    \begin{column}{.7\textwidth}
      \begin{figure}
        \includegraphics[width=\textwidth]{subplotHistNettoyage.pdf}
      \end{figure}
    \end{column}
  \end{columns}
\end{frame}

\begin{frame}{\insertsubsection}
  \begin{columns}
    \begin{column}{.3\textwidth}
      \begin{itemize}
        \item On obtient des distributions plus intéressantes
        \item[]
        \item Fonctionnement assez similaire des différents types d'énergies
      \end{itemize}
    \end{column}
    \begin{column}{.7\textwidth}
      \begin{figure}
        \includegraphics[width=\textwidth]{subplotHistQ99.pdf}
      \end{figure}
    \end{column}
  \end{columns}
\end{frame}

\subsection{Données redondantes}
\begin{frame}{\insertsubsection}
  \begin{columns}
    \begin{column}{.4\textwidth}
      \begin{itemize}
        \item Mêmes valeurs energy\_100g et energy-kj\_100g
        \item[]
        \item Variation pour ce qui est de l'energie en kcal pour 100g
        \item[]
        \item Nous n'allons garder que les colonnes avec des unités d'énergie : kcal et kj
        \item[]
        \item De même nous allons supprimer les colonnes finissant par \_tags et \_en
              qui contiennent des données redondantes des colonnes du même nom sans suffixes
      \end{itemize}
    \end{column}
    \begin{column}{.6\textwidth}
      \begin{figure}
        \includegraphics[width=.9\textwidth]{EnergyReg.pdf}
      \end{figure}
    \end{column}
  \end{columns}
\end{frame}

\section{Exploration du jeu de données}
\subsection{Analyse univariée}
\begin{frame}{\insertsubsection}
  \begin{table}
    \tiny
    \input{./Tableaux/StatsNum1}
    \input{./Tableaux/StatsNum2}
    \input{./Tableaux/StatsNum3}
    \caption{Tableaux des statistiques sur chaques variables numériques}
  \end{table}
\end{frame}

\begin{frame}{\insertsubsection}
  \begin{figure}
    \includegraphics[height=.8\textheight]{subplotBox}
  \end{figure}
\end{frame}

\subsection{Analyse bivariée}

\section{Idée d'application}
\begin{frame}
  \begin{itemize}
    \item Pouvoir scanner un produit et trouver des alternatives similaires de meilleure qualité
          \begin{itemize}
            \item Meilleures pour la santé (nutriscore plus élevé)
                  \begin{itemize}
                    \item Avec moins d'additifs (nombre d'additifs inférieurs)
                  \end{itemize}
            \item Meilleures pour l'environement (ecoscore plus élevé)
                  \begin{itemize}
                    \item Origine plus proche (France, Europe)
                  \end{itemize}
            \item Voir les deux
          \end{itemize}
    \item[]
    \item Pouvoir choisir selon un régime alimentaire particulier
          \begin{itemize}
            \item Végétarien
            \item Vegan
            \item Halal
            \item Kasher
            \item Sans gluten
            \item etc
          \end{itemize}
    \item[]
    \item[] Implémenter un système de reconnaissance d'image
      permettant d'extraire les données manquantes des photos des étiquettes
  \end{itemize}
\end{frame}

\section{Conclusion}

\end{document}