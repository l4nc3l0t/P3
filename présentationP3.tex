% !TEX root = PSanté_03_présentation.tex

\usefonttheme{serif}
\usepackage{fontspec}
	\setmainfont{TeX Gyre Heros}
\usepackage{unicode-math}
\usepackage{lualatex-math}
	\setmathfont{TeX Gyre Termes Math}
\usepackage{polyglossia}
\setdefaultlanguage[frenchpart=false]{french}
\setotherlanguage{english}
\usepackage{microtype}
\usepackage[locale = FR,
            separate-uncertainty,
            multi-part-units = single,
            range-units = single]{siunitx}
	\DeclareSIUnit\an{an}
  \DeclareSIUnit{\octet}{o}
\usepackage{amsmath}
\usepackage{amsfonts}
\usepackage{amssymb}
\usepackage{array}
\usepackage{graphicx}
\graphicspath{{./Figures/}}
\usepackage{booktabs}
\usepackage{tabularx}
\usepackage{multirow}
\usepackage{multicol}
    \newcolumntype{L}{>{\raggedright\arraybackslash}X}
    \newcolumntype{R}{>{\raggedleft\arraybackslash}X}
\usepackage{tikz}
\usetikzlibrary{mindmap}
\usetikzlibrary{overlay-beamer-styles}
\usepackage{subcaption}
\usepackage[]{animate}
\usepackage{float}
\usepackage{csquotes}

\usetheme[secheader
         ]{Boadilla}
\usecolortheme{seagull}
\setbeamertemplate{enumerate items}[default]
\setbeamertemplate{itemize items}{-}
\setbeamertemplate{navigation symbols}{}
\setbeamertemplate{bibliography item}{}
\setbeamerfont{framesubtitle}{size=\large}
\setbeamertemplate{section in toc}[sections numbered]
%\setbeamertemplate{subsection in toc}[subsections numbered]

\title[Concevez une application au service de la santé publique]{Projet 3 : Concevez une application au service de la santé publique}
\author[Lancelot \textsc{Leclercq}]{Lancelot \textsc{Leclercq}} 
\institute[]{}
\date[]{\small{2 novembre 2021}}

%\AtBeginSection[]{
%  \begin{frame}
%  \vfill
%  \centering
%    \usebeamerfont{title}\insertsectionhead\par%
%  \vfill
%  \end{frame}
%}

\begin{document}
\setbeamercolor{background canvas}{bg=gray!20}
\begin{frame}[plain]
  \titlepage
\end{frame}

\section{Introduction}

\subsection{Problématique}
\begin{frame}{\insertsubsection}
  \begin{columns}
    \begin{column}{.6\textwidth}
      \begin{itemize}
        \item Application au service de la santé publique
        \item[]
        \item Idées d'application en lien avec l'alimentation
      \end{itemize}
    \end{column}
    \begin{column}{.4\textwidth}
      \includegraphics[width=.8\textwidth]{./Sante-publique-France-logo.pdf}
    \end{column}
  \end{columns}
\end{frame}

\subsection{Jeu de données}
\begin{frame}{\insertsubsection}
  \begin{columns}
    \begin{column}{.6\textwidth}
      \begin{itemize}
        \item Jeu de données issu de Open Food Facts
        \item[]
        \item Base de donnée ouverte et participative
        \item[]
        \item Repertorie les ingrédients, les allergènes, composition nutritionnelle, les labels, l'origine, etc
        \item[]
        \item Grosse base de données (environ \qty{4.5}{\giga\octet})
      \end{itemize}
    \end{column}
    \begin{column}{.4\textwidth}
      \centering
      \includegraphics[width=.8\textwidth]{./openfoodfacts-logo-en.pdf}
    \end{column}
  \end{columns}
\end{frame}


\section{Nettoyage du jeu de données}
\subsection{Produits vendus en France}
\begin{frame}{\insertsubsection}
  \begin{columns}[T]
    \begin{column}{.4\textwidth}
      \begin{itemize}
        \item Sélection des produits vendus en France
        \item \num{850000} entrées sur les 2 milliards d'origine
      \end{itemize}
    \end{column}
    \begin{column}{.6\textwidth}
      \begin{itemize}
        \item Suppression des colonnes dans lesquelles on a moins de 1\% de données
              et des colonnes contenant des tags et des versions anglaises qui sont redondantes avec les
              colonnes du mêmes nom
        \item 57 colonnes restantes sur les 186 d'origine
      \end{itemize}
    \end{column}
  \end{columns}
  \vfill
  \begin{figure}
    \includegraphics[width=.75\textwidth]{DataDropFinal.pdf}
  \end{figure}
\end{frame}

\subsection{Nettoyage des outliers}
\begin{frame}{\insertsubsection}
  \begin{columns}
    \begin{column}{.3\textwidth}
      \begin{itemize}
        \item Valeurs extrèmes pour :
              \begin{itemize}
                \item Quantité servie
                \item Empreinte carbone
                \item Énergie
              \end{itemize}
        \item[]
        \item Des valeurs très importantes là où le nombre de valeurs est très faible
        \item[]
        \item Nous écartons ces valeurs lorsqu'elles sont en dehors du quartile qui contient 99\% des valeurs.
      \end{itemize}
    \end{column}
    \begin{column}{.7\textwidth}
      \begin{figure}
        \includegraphics[width=\textwidth]{subplotHistNettoyage.pdf}
      \end{figure}
    \end{column}
  \end{columns}
\end{frame}

\begin{frame}{\insertsubsection}
  \begin{columns}
    \begin{column}{.3\textwidth}
      \begin{itemize}
        \item On obtient des distributions plus intéressantes
        \item[]
        \item Fonctionnement assez similaire des différents types d'énergies
      \end{itemize}
    \end{column}
    \begin{column}{.7\textwidth}
      \begin{figure}
        \includegraphics[width=\textwidth]{subplotHistQ99.pdf}
      \end{figure}
    \end{column}
  \end{columns}
\end{frame}

\subsection{Données redondantes}
\begin{frame}{\insertsubsection}
  \begin{columns}
    \begin{column}{.4\textwidth}
      \begin{itemize}
        \item Mêmes valeurs energy\_100g et energy-kj\_100g
        \item[]
        \item Variation pour ce qui est de l'energie en kcal pour 100g
        \item[]
        \item Nous n'allons garder que les colonnes avec des unités d'énergie : kcal et kj
        \item[]
        \item De même nous allons supprimer les colonnes finissant par \_tags et \_en
              qui contiennent des données redondantes des colonnes du même nom sans suffixes
      \end{itemize}
    \end{column}
    \begin{column}{.6\textwidth}
      \begin{figure}
        \includegraphics[width=.9\textwidth]{EnergyReg.pdf}
      \end{figure}
    \end{column}
  \end{columns}
\end{frame}

\section{Exploration du jeu de données}
\subsection{Analyse univariée}
\begin{frame}{\insertsubsection}
  \begin{table}
    \tiny
    \input{./Tableaux/StatsNum1.tex}
    \input{./Tableaux/StatsNum2.tex}
    \input{./Tableaux/StatsNum3.tex}
    \caption{Tableaux des statistiques sur chaques variables numériques}
  \end{table}
\end{frame}

\begin{frame}{\insertsubsection}
  \begin{figure}
    \includegraphics[height=.8\textheight]{subplotBox}
  \end{figure}
\end{frame}

\begin{frame}{\insertsubsection}
  \begin{columns}
    \begin{column}{.5\textwidth}
      \begin{itemize}
        \item Observation du nombre d'occurences de chaques catégories de nutriscore
        \item[]
        \item Diagramme circulaire permet de visualiser les proportions
      \end{itemize}
    \end{column}
    \begin{column}{.5\textwidth}
      \begin{figure}
        \includegraphics[width=\textwidth]{NutriPie.pdf}
      \end{figure}
    \end{column}
  \end{columns}
\end{frame}

\subsection{Analyse bivariée}
\begin{frame}{\insertsubsection}
  \begin{columns}
    \begin{column}{.55\textwidth}
      \begin{itemize}
        \item Les catégories du nutriscore semblent corrélées avec le score
        \item[]
        \item Vérification par ANOVA à une dimension
        \item H0 les moyennes des différentes catégories ne varient pas en fonction du score
      \end{itemize}
      \begin{table}
        \tiny
        \input{./Tableaux/ANOVANutriScoreGrade.tex}
        \caption{Résultats de l'ANOVA à une dimension entre le score du nutriscore et ses catégories}
      \end{table}
      \begin{itemize}
        \item p-value \qty{< 5}{\percent} on rejette l'hypothèse pour chaque catégories
      \end{itemize}
    \end{column}
    \begin{column}{.45\textwidth}
      \begin{figure}
        \includegraphics[width=\textwidth]{NutriBox.pdf}
      \end{figure}
    \end{column}
  \end{columns}
\end{frame}

\subsubsection{Régressions}
\begin{frame}{\insertsubsubsection}
  \begin{columns}
    \begin{column}{.5\textwidth}
      \begin{itemize}
        \item La matrice des corrélations nous permet d'observer quelles variables
              semblent fonctionner de façon similaire
        \item[]
        \item Nous allons faire quelques regressions entre ces variables proches
      \end{itemize}
    \end{column}
    \begin{column}{.5\textwidth}
      \includegraphics[width=.98\textwidth]{HeatmapNum.pdf}
    \end{column}
  \end{columns}
\end{frame}

\begin{frame}{\insertsubsubsection}
  \begin{columns}
    \begin{column}{.5\textwidth}
      \begin{figure}
        \includegraphics[width=\textwidth]{SelReg.pdf}
        \includegraphics[width=\textwidth]{GrasReg.pdf}
      \end{figure}
    \end{column}
    \begin{column}{.5\textwidth}
      \begin{figure}
        \includegraphics[width=\textwidth]{SucreReg.pdf}
        \includegraphics[width=\textwidth]{EnergyFatReg.pdf}
      \end{figure}
    \end{column}
  \end{columns}
\end{frame}

\subsection{Analyse Multivariée}
\begin{frame}{\insertsubsection}
  \begin{columns}
    \begin{column}{.5\textwidth}
      \begin{itemize}
        \item Régression multiple entre le nutriscore et les variables
              sucre, gras, sel et energie
        \item[]
        \item H0 les moyennes des variables sont égales entre elles
              ainsi qu'avec celle du nutriscore
        \item[]
        \item p-value \qty{< 5}{\percent} les variables ne varient donc pas très similairement
        \item[]
        \item Nous allons tout de même réaliser quelques graphiques des régressions
              entre Nutriscore et chaques variables
      \end{itemize}
    \end{column}
    \begin{column}{.5\textwidth}
      \begin{table}
        \tiny
        \input{./Tableaux/MultiReg}
      \end{table}
    \end{column}
  \end{columns}
\end{frame}

\subsubsection{Régressions avec le nutriscore}
\begin{frame}{\insertsubsubsection}
  \begin{columns}
    \begin{column}{.5\textwidth}
      \begin{figure}
        \includegraphics[width=\textwidth]{NutriEnergyReg.pdf}
        \includegraphics[width=\textwidth]{NutriSucreReg.pdf}
      \end{figure}
    \end{column}
    \begin{column}{.5\textwidth}
      \begin{figure}
        \includegraphics[width=\textwidth]{NutriGrasReg.pdf}
        \includegraphics[width=\textwidth]{NutriSelReg.pdf}
      \end{figure}
    \end{column}
  \end{columns}
\end{frame}

\subsubsection{ACP}
\begin{frame}{\insertsubsubsection}
  \begin{columns}
    \begin{column}{.5\textwidth}
      \begin{itemize}
        \item L'ACP permet de regrouper des variables qui fonctionnent de manière similaire
        \item[]
        \item ACP sur les colonnes numériques
        \item[]
        \item Le graphique des variances cumulées nous montre qu'il n'y a pas de rupture nette ("coude")
              dans la part expliquée des différentes composantes
      \end{itemize}
    \end{column}
    \begin{column}{.5\textwidth}
      \includegraphics[width=\textwidth]{ScreePlot.pdf}
    \end{column}
  \end{columns}
\end{frame}

\begin{frame}
  \begin{columns}
    \begin{column}{.5\textwidth}
      \includegraphics[width=\textwidth]{PCAF1F2.pdf}
    \end{column}
    \begin{column}{.5\textwidth}
      \includegraphics[width=\textwidth]{PCAF1F3.pdf}
    \end{column}
  \end{columns}
\end{frame}

\begin{frame}
  \begin{columns}
    \begin{column}{.5\textwidth}
      \includegraphics[width=\textwidth]{PCAF1F4.pdf}
    \end{column}
    \begin{column}{.5\textwidth}
      \includegraphics[width=\textwidth]{PCAF2F3.pdf}
    \end{column}
  \end{columns}
\end{frame}

\begin{frame}
  \begin{columns}
    \begin{column}{.5\textwidth}
      \includegraphics[width=\textwidth]{PCAF2F4.pdf}
    \end{column}
    \begin{column}{.5\textwidth}
      \includegraphics[width=\textwidth]{PCAF3F4.pdf}
    \end{column}
  \end{columns}
\end{frame}

\section{Idée d'application}
\begin{frame}
  \begin{itemize}
    \item Pouvoir scanner un produit et trouver des alternatives similaires de meilleure qualité
          \begin{itemize}
            \item Meilleures pour la santé (nutriscore plus élevé)
                  \begin{itemize}
                    \item Avec moins d'additifs (nombre d'additifs inférieurs)
                  \end{itemize}
            \item Meilleures pour l'environement (ecoscore plus élevé)
                  \begin{itemize}
                    \item Origine plus proche (France, Europe)
                  \end{itemize}
            \item Voir les deux
          \end{itemize}
    \item[]
    \item Pouvoir choisir selon un régime alimentaire particulier
          \begin{itemize}
            \item Végétarien
            \item Vegan
            \item Halal
            \item Kasher
            \item Sans gluten
            \item etc
          \end{itemize}
    \item[]
    \item Implémenter un système de reconnaissance d'image
          permettant d'extraire les données manquantes des photos des étiquettes
    \item[] 
    \item Pouvoir renseigner le prix de l'aliment et permettre de comparer le prix 
          entre les aliments selon la qualité
    \begin{itemize}
      \item[] Comparer le prix d'un panier 1er prix avec celui d'un panier composé de produits
           de meilleure qualité afin de permettre aux ménages de voir la 
          différence entre des paniers de qualité différentes et pouvoir choisir
          quel type d'alimentation il veulent en fonction de la part du pouvoir d'achat 
          qu'ils sont prêt à mettre dans l'alimentation
    \end{itemize}
  \end{itemize}
\end{frame}

\section{Conclusion}

\end{document}